\documentclass[11pt]{article}

% strange formatting added last minute to remove warnings
\headheight = 14pt

% packages
\usepackage{physics}
% margin spacing
\usepackage[top=1in, bottom=1in, left=0.5in, right=0.5in]{geometry}
\usepackage{hanging}
\usepackage{amsfonts, amsmath, amssymb, amsthm}
\usepackage{systeme}
\usepackage[none]{hyphenat}
\usepackage{fancyhdr}
\usepackage[nottoc, notlot, notlof]{tocbibind}
\usepackage{graphicx}
\graphicspath{{./images/}}
\usepackage{float}
\usepackage{siunitx}
\usepackage{esint}
\usepackage{cancel}

% colors
\usepackage{xcolor}
\definecolor{p}{HTML}{FFDDDD}
\definecolor{g}{HTML}{D9FFDF}
\definecolor{y}{HTML}{FFFFCF}
\definecolor{b}{HTML}{D9FFFF}
\definecolor{o}{HTML}{FADECB}
%\definecolor{}{HTML}{}

% \highlight[<color>]{<stuff>}
\newcommand{\highlight}[2][p]{\mathchoice%
  {\colorbox{#1}{$\displaystyle#2$}}%
  {\colorbox{#1}{$\textstyle#2$}}%
  {\colorbox{#1}{$\scriptstyle#2$}}%
  {\colorbox{#1}{$\scriptscriptstyle#2$}}}%

% header/footer formatting
\pagestyle{fancy}
\fancyhead{}
\fancyfoot{}
\fancyhead[L]{COP3530 Prof. Kapoor}
\fancyhead[C]{Project 3 Documentation}
\fancyhead[R]{Sai Sivakumar, Will McCoy}
\fancyfoot[R]{\thepage}
% remove underlined header
%\renewcommand{\headrulewidth}{0pt}

% paragraph indentation/spacing
\setlength{\parindent}{0cm}
\setlength{\parskip}{5pt}
\renewcommand{\baselinestretch}{1.25}

% bracketing macro
\newcommand{\br}[1]{\left(#1\right)}
\newcommand{\sbr}[1]{\left[#1\right]}
\newcommand{\cbr}[1]{\left\{#1\right\}}

% set page count index to begin from 1
\setcounter{page}{1}

\begin{document}
\section{This is a section.}
This is text, but this is \textit{inline} math: $2\exp(x), x\in\sbr{a,b}$. Observe that things which are usually taller are shrunk, like $\frac{1}{2}, \sum_k, \mathcal{E}\br{1-e^{\frac{-t}{RC}}}, \int_a^b f(x)\dd{x}$.

This is more text, but we can follow with \textit{display mode} math that looks like \[\int_{\gamma}f(z)\dd{z} = 2\pi i \sum_k\Res(f,a_k) ,\] and we end the sentence here.

\subsection{Test subsection}
If we have a rather \textbf{large} expression for whatever reason, we can try splitting it over multiple lines.

Observe that we can split \begin{multline*}
    c_1\cos(4\log(x)) + c_2\sin(4\log(x)) -\br{1+\frac{1}{2}+\frac{1}{4}+\cdots+\frac{1}{2^k}} \\ +\exp(\int p(x)\dd{x}) + \eval{uv}_a^b - \int_a^b v\dd{u} + \lim_{\beta \to \infty}\frac{1}{2\pi i}\int_{\gamma - i\beta}^{\gamma + i\beta}e^{zt}f(z)\dd{z}
\end{multline*} like so.

We may even align several equations, at the \texttt{\&} symbol placed before the $=$ sign. \begin{align*}
    c_1x+c_2y &= 0 \\
    c_1x-c_2y &= 0
\end{align*}

\section{to conclude, one last item}
I added a package which allows us to $\highlight{\text{highlight}}$ (default is pink) math. Here the math is actually text since I marked it as text, but we can highlight math in other colors; for example, \[\highlight[b]{O\br{f(n)-n^{7/6}}}.\]

For now we can leave this here but when we begin documentation after everything is said and done, remove everything above this line. I am unsure what components above will be useful in our documentation but otherwise it should be good.
\end{document}